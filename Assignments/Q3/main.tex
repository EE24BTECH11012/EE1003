\let\negmedspace\undefined
\let\negthickspace\undefined
\documentclass[journal]{IEEEtran}
\usepackage[a5paper, margin=10mm, onecolumn]{geometry}
%\usepackage{lmodern} % Ensure lmodern is loaded for pdflatex
\usepackage{tfrupee} % Include tfrupee package

\setlength{\headheight}{1cm} % Set the height of the header box
\setlength{\headsep}{0mm}     % Set the distance between the header box and the top of the text

\usepackage{gvv-book}
\usepackage{gvv}
\usepackage{cite}
\usepackage{amsmath,amssymb,amsfonts,amsthm}
\usepackage{algorithmic}
\usepackage{graphicx}
\usepackage{textcomp}
\usepackage{xcolor}
\usepackage{txfonts}
\usepackage{listings}
\usepackage{enumitem}
\usepackage{mathtools}
\usepackage{gensymb}
\usepackage{comment}
\usepackage[breaklinks=true]{hyperref}
\usepackage{tkz-euclide} 
\usepackage{listings}
% \usepackage{gvv}                                        
\def\inputGnumericTable{}                                 
\usepackage[latin1]{inputenc}                                
\usepackage{color}                                            
\usepackage{array}                                            
\usepackage{longtable}                                       
\usepackage{calc}                                             
\usepackage{multirow}                                         
\usepackage{hhline}                                           
\usepackage{ifthen}                                           
\usepackage{lscape}
\begin{document}

\bibliographystyle{IEEEtran}
\vspace{3cm}

\title{8.3.12}
\author{EE24BTECH11012 - Bhavanisankar G S}
% \maketitle
% \newpage
% \bigskip
{\let\newpage\relax\maketitle}

\renewcommand{\thefigure}{\theenumi}
\renewcommand{\thetable}{\theenumi}
\setlength{\intextsep}{10pt} % Space between text and floats


\numberwithin{equation}{enumi}
\numberwithin{figure}{enumi}
\renewcommand{\thetable}{\theenumi}


\section{\textbf{SOLVING SECOND ORDER DIFFERENTIAL EQUATION BY THE METHOD OF ANNIHILATORS}}
\begin{itemize}
\item The method of solving higher order non-homogeneous differential equation
$$ L[y] = g(t) $$ by homogenizing it, i.e., by making the $g(t) = 0$ is called \color{blue} Annihilator method \color{black}.
\item This can be done by subsequent differentiation on both sides and by appropriate scaling.
\item \textbf{NOTATION : } Consider the differential equation, 
\begin{equation}
	L[y] = g(t)
\end{equation}
and the \color{blue} Differential operator \color{black}
\begin{equation}
	D = \frac{d}{dt} g(t) \\
	D^n = \frac{d^n}{dt^n} g(t)
\end{equation}
\end{itemize}
This can be seen with the following example - \\
Consider $g(t) = \sin{at}$ \\
\begin{align*}
	g(t) &= \sin{at} \\
	D &= a \cos{at} \\
	D^2 &= -a^2 \sin{at} \\
	\implies D^2 + a^2 = 0
\end{align*}
is the required annihilator of $L[y] = \sin{at}$.
\begin{table}[htbp]
\centering
\caption{Functions and Their Annihilators}
\begin{tabular}{|l|l|}
\hline
\textbf{Function $f(x)$} & \textbf{Annihilator $Q(D)$} \\
\hline
$a_n x^n + a_{n-1} x^{n-1} + \cdots + a_1 x + a_0$ & $D^{n+1}$ \\
\hline
$e^{ax}$ & $D - a$ \\
\hline
$x^n e^{ax}$ & $(D - a)^{n+1}$ \\
\hline
$\cos(ax)$ or $\sin(ax)$ & $D^2 + a^2$ \\
\hline
$x^n \cos(ax)$ or $x^n \sin(ax)$ & $(D^2 + a^2)^{n+1}$ \\
\hline
$e^{ax} \cos(bx)$ or $e^{ax} \sin(bx)$ & $(D-a)^2 + b^2$ \\
\hline
$x^n e^{ax} \cos(bx)$ or $x^n e^{ax} \sin(bx)$ & $\left[(D-a)^2 + b^2\right]^{n+1}$ \\
\hline
$a_n x^n + \cdots + a_0 + b_1 e^{c_1 x} + \cdots + b_m e^{c_m x}$ & $D^{n+1}(D-c_1) \cdots (D-c_m)$ \\
\hline
\end{tabular}
\end{table}

\begin{table}[htbp]
\centering
\caption{Additional Notes on Annihilators}
\begin{tabular}{|l|p{10cm}|}
\hline
\textbf{Property} & \textbf{Description} \\
\hline
Sum Rule & If $f(x) = g(x) + h(x)$, then an annihilator for $f(x)$ is $A(D) = A_g(D) \cdot A_h(D)$ where $A_g(D)$ annihilates $g(x)$ and $A_h(D)$ annihilates $h(x)$ \\
\hline
Minimal Annihilator & Always use the smallest degree annihilator possible to minimize the order of the resulting homogeneous equation \\
\hline
Constant Multiples & If $A(D)$ annihilates $f(x)$, then $A(D)$ also annihilates $c \cdot f(x)$ for any constant $c$ \\
\hline
Application Method & Apply the annihilator to both sides of the equation $L[y] = g(x)$ to get a homogeneous equation $A(D)L[y] = 0$ \\
\hline
\end{tabular}
\end{table}

\section{\textbf{SOLVING THE GIVEN CIRCUIT} : }
\section{Response of RL series circuit with square wave input}
Consider a series RL circuit as shown in Figure \ref{fig:rl}
% Circuit diagram
\begin{figure}[!ht]
\centering
\resizebox{1\textwidth}{!}{%
\begin{circuitikz}
\tikzstyle{every node}=[font=\normalsize]
\draw (5.5,18) to[square voltage source, sources/symbol/rotate=auto] (5.5,15);
\draw (5.5,18) to[short] (6.5,18);
\draw (6.5,18) to[R] (9,18);
\draw (9,18) to[L ] (9,15);
\draw (5.5,15) to[short] (9,15);
\node [font=\normalsize] at (5,16.5) {V};
\node [font=\normalsize] at (7.5,18.5) {R};
\node [font=\normalsize] at (8.5,16.5) {L};
\end{circuitikz}
}%
\label{fig:rl}
\end{figure}

% Square wave input
\begin{figure}[!ht]
\centering
\resizebox{1\textwidth}{!}{%
\begin{circuitikz}
\tikzstyle{every node}=[font=\normalsize]
\draw [->, >=Stealth] (5.75,14) -- (5.75,18.5);
\draw [->, >=Stealth] (5.5,14.5) -- (14.25,14.5);
\draw (5.75,17) to[short] (8.25,17);
\draw (8.25,17) to[short] (8.25,14.5);
\draw (8.25,14.5) to[short] (9.25,14.5);
\draw (9.25,14.5) to[short] (12.5,14.5);
\draw (9.25,14.5) to[short] (9.25,17);
\draw (9.25,17) to[short] (11.75,17);
\draw (11.75,17) to[short] (11.75,14.5);
\draw (11.5,14.5) to[short] (12.5,14.5);
\draw (12.75,14.5) to[short] (12.75,17);
\draw (12.75,17) to[short] (13.5,17);
\draw [dashed] (5.75,17) -- (14.25,17);
\node [font=\normalsize] at (14.5,14.5) {t};
\node [font=\normalsize] at (5.25,18.25) {v(t)};
\node [font=\normalsize] at (5.5,17) {10};
\node [font=\normalsize] at (8.25,14.25) {$\alpha$ T};
\node [font=\normalsize] at (9.25,14.25) {T};
\node [font=\normalsize] at (11.5,14.25) {(1+$\alpha$) T};
\node [font=\normalsize] at (12.75,14.25) {2T};
\end{circuitikz}
}%
\label{fig:input}
\end{figure}

The governing differential equation of the scenario can be written as - 
\begin{equation}
	\frac{di}{dt} + \brak{\frac{R}{L}} i = \frac{V}{L} \label{eq:rlresp}
\end{equation}
The above equation \eqref{eq:rlresp} can be solved analytically by decomposing the input as sum of sines and cosines of different frequencies, treating each term as individual voltage sources, computing their individual responses and clubbing them as the overall response of the circuit.

\subsection{\textbf{Response of $L[y] = a_0$} : }
Consider a series RL circuit with constant voltage source. The governing DE can be framed as \\
\begin{align*}
	\frac{di}{dt} + \brak{\frac{R}{L}} i &= \frac{a_{0}}{L} \\
	\frac{di}{dt} &= \frac{a_0 - iR}{L} \\
	\frac{di}{a_0 - iR} &= \frac{dt}{L} \\
	\frac{-1}{R} \log{a_0 - iR} &= \frac{t}{L} \\
	i_1 &= \frac{a_0}{R} - c_1 e^{-\frac{tR}{L}} \label{eq:i1}
\end{align*}

\subsection{\textbf{Response of $L[y] = a_{k} \cos{k \omega_{0} t}$} : }
Consider a series RL circuit with cosinusoidal voltage source. The governing DE can be framed as:
\begin{align*}
	\frac{di}{dt} + \left(\frac{R}{L}\right) i &= \frac{a_{k} \cos{k \omega_{0} t}}{L} \\
	i \left(e^{\frac{Rt}{L}}\right) &= \int \left(e^{\frac{Rt}{L}}\right) \frac{a_{k} \cos{k \omega_{0} t}}{L} \, dt \nonumber \\
	i_2 &= c_2 e^{-\frac{Rt}{L}} + \frac{a_{k}}{R^2 + (k \omega_{0} L)^2} \left(R \cos{k \omega_{0} t} + k \omega_{0} L \sin{k \omega_{0} t}\right) \label{eq:i2}
\end{align*}

\subsection{\textbf{Response of $L[y] = b_{k} \sin{k \omega_{0} t}$} : }
Consider a series RL circuit with sinusoidal voltage source. The governing DE can be framed as:
\begin{align*}
	\frac{di}{dt} + \left(\frac{R}{L}\right) i &= \frac{b_{k} \sin{k \omega_{0} t}}{L} \\
	i \left(e^{\frac{Rt}{L}}\right) &= \int \left(e^{\frac{Rt}{L}}\right) \frac{b_{k} \sin{k \omega_{0} t}}{L} \, dt \nonumber \\
	i_3 &= c_3 e^{-\frac{Rt}{L}} + \frac{b_{k}}{R^2 + (k \omega_{0} L)^2} \left(R \sin{k \omega_{0} t} - k \omega_{0} L \cos{k \omega_{0} t}\right) \label{eq:i3}
\end{align*}


\section{\textbf{FINDING FOURIER COEFFICIENTS} : }
By the method of \color{blue} Fourier analysis \color{black}, it is proposed that \textbf{" } \textit{Every periodic function can be written as a sum of sines and cosines of different frequencies.} \textbf{ "}. Some important properties related to this are - 
\begin{enumerate}
\item \textbf{Orthogonality : } - The integral over a period of the product of two sines or cosines with different frequencies evaluates to $0$. Hence, sines and cosines of different frequencies are always orthogonal over a period.
\begin{align*}
	\int_{a}^{a + nT} \sin{mt} \sin{nt} &= 0, m \neq n \\
	\int_{a}^{a + nT} \sin{mt} \cos{nt} &= 0, m \neq n
\end{align*}
\item \textbf{Projection : } - Projection of a function $f(t)$ on $g(t)$ in a interval $[a,b]$ is given as - 
\begin{align*}
	P(t) &= \int_{a}^{b} f(t) g(t) dt
\end{align*}
\end{enumerate}
Consider a function $g(t)$. By the method of Fourier analysis, we have
\begin{equation}	
	g(t) = a_{0} + \sum_{k = 1}^{\infty} \brak{a_{k} \cos{k \omega_{0} t} + b_{k} \sin{k \omega_{0} t}}
\end{equation}
The individual coefficients, $a_{k} \brak{\text{ or } b_{k}}$  can be found out by multiplying $\cos{k \omega_{0} t} \brak{\text{ or } \sin_{k \omega_{0} t}}$ and integrating over a period.
\begin{align*}
	\int_{0}^{2 \pi} g(t) \cos{k \omega_{0} t} dt &= \int_{0}^{2 \pi} \brak{a_{0} + \sum_{k = 1}^{\infty} \brak{a_{k} \cos{k \omega_{0} t} + b_{k} \sin{k \omega_{0} t}}} dt \\
	\int_{0}^{2 \pi} g(t) \cos{k \omega_{0} t} dt &= a_{k} \int_{0}^{2 \pi} \cos^{2}{k \omega_{0} t} dt \\
	\int_{0}^{2 \pi} g(t) \cos{k \omega_{0} t} dt &= a_{k} \pi \\
	a_{k} &= \frac{1}{\pi} \brak{\int_{0}^{2 \pi} g(t) \cos{k \omega_{0} t}} dt \\
	a_{0} &= \frac{1}{2 \pi} \brak{\int_{0}^{2 \pi} g(t)} dt
\end{align*}
Similarly, 
\begin{align*}
	b_{k} &= \frac{1}{\pi} \brak{\int_{0}^{2 \pi} g(t) \sin{k \omega_{0} t}} dt
\end{align*}

\subsection{\textbf{Finding the coefficients of the square wave } : }
Consider the equation, 
\begin{equation}
g(t) = 
\begin{cases} 
A, & 0 \leq t < \alpha T \\ 
0, & \alpha T \leq t < T
\end{cases}
\end{equation}
Determining the coefficients, we have
\begin{align*}
	a_{0} &= \frac{1}{2 \pi} \brak{\int_{0}^{2 \pi} g(t)} dt \\
	a_{0} &= \frac{1}{T} \brak{\int_{0}^{\alpha T} A} dt \\
	a_{0} &= \alpha A \label{eq:a0}
\end{align*}
Similarly,
\begin{align*}
	a_{k} &= \frac{1}{\pi} \brak{\int_{0}^{2 \pi} g(t) \cos{k \omega_{0} t}} dt \\
	a_{k} &= \frac{2}{T} \brak{\int_{0}^{\alpha T} A \cos{k \omega_{0} t}} dt \\
	a_{k} &= \frac{A}{\pi k} \brak{\sin{2 \pi k \alpha}} \label{eq:ak} \\
	b_{k} &= \frac{1}{\pi} \brak{\int_{0}^{2 \pi} g(t) \sin{k \omega_{0} t}} dt \\
	b_{k} &= \frac{2}{T} \brak{\int_{0}^{\alpha T} A \sin{ k \omega_{0} t} } dt \\
	b_{k} &= \frac{A}{\pi k} \brak{1 - \cos{2 \pi k \alpha}} \label{eq:bk}
\end{align*}

\section{\textbf{SOLVING RL CIRCUIT USING FOURIER ANALYSIS} : }
From \eqref{eq:i1}, \eqref{eq:i2} and \eqref{eq:i3}, we have the net current, $i$ as
\begin{equation}
	i = i_1 + i_2 + i_3
\end{equation}
\begin{align*}
	i &= \frac{a_0}{R} - c_1 e^{-\frac{tR}{L}} + \sum_{k = 1}^{\infty} \brak{c_2 e^{-\frac{Rt}{L}} + \frac{a_{k}}{R^2 + (k \omega_{0} L)^2} \brak{R \cos{k \omega_{0} t} + k \omega_{0} L \sin{k \omega_{0} t}} + \\ & c_3 e^{-\frac{Rt}{L}} + \frac{a_{k}}{R^2 + (k \omega_{0} L)^2} \brak{R \sin{k \omega_{0} t} - k \omega_{0} L \cos{k \omega_{0} t}}} \\
	i &= c e^{-\frac{Rt}{L}} + \frac{a_0}{R} + \sum_{k=1}^{\infty} \sbrak{\brak{\frac{a_{k}R - b_{k}k \omega_{0}L}{R^2 + (k \omega_{0} L)^2}} \cos{k \omega_{0} t} + \brak{\frac{a_{k}k \omega_{0} L + b_{k}R}{R^2 + (k \omega_{0} L)^2}} \sin{k \omega_{0} t}} \\
	i &= c e^{-\frac{Rt}{L}} + \frac{\alpha A}{R} + \sum_{k=1}^{\infty} \sbrak{\brak{\frac{A}{\pi k} \brak{\frac{R \sin{2 \pi k \alpha} + k \omega_{0} L \cos{2 \pi k \alpha} - k \omega_{0} L}{R^2 + (k \omega_{0} L)^2}}} \cos{(k \omega_{0} t)} + \\ & \brak{\frac{A}{\pi k} \brak{\frac{k \omega_{0} L \sin{2 \pi k \alpha} - R \cos{2 \pi k \alpha} + R}{R^2 + (k \omega_{0} L)^2}}} \sin{(k \omega_{0} t)}}
\end{align*}


\end{document}
